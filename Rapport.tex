\documentclass{myRapportECL}
\usepackage{lipsum}
\usepackage{listings} 
\usepackage{bigstrut}
\title{} %Titre du fichier

\begin{document}

\lstset{numbers=left, 
	basicstyle=\small\sffamily,
	numbers=left,
 	numberstyle=\tiny,
	frame=none,
	tabsize=4,
	columns=fixed,
	showstringspaces=false,
	showtabs=false,
	keepspaces,
	commentstyle=\color{red},
	keywordstyle=\color{blue},
    breaklines=true,
    backgroundcolor=\color{lightgray!40!white},
    basicstyle=\ttfamily,
    commentstyle=\ttfamily\color{green!40!black}
} 

%----------- Informations du rapport ---------

\titre{} %Titre du fichier .pdf
\UE{} %Nom de la UE
\sujet{} %Nom du sujet

\enseignant{Prénom \textsc{Nom}} %Nom de l'enseignant

\eleves{Prénom \textsc{NOM} \\
		Prénom \textsc{NOM} } %Nom des élèves

%----------- Initialisation -------------------
        
\fairemarges %Afficher les marges
\fairepagedegarde %Créer la page de garde
\tabledematieres %Créer la table de matières

%------------ Corps du rapport ----------------
\centerline{\textbf{\huge{"Réponse uniquement d'une base de données"}}}

\section{Introduction}

\section{Manipulation}


\section{Conclusion}
\end{document}

% \begin{lstlisting}[language={prolog}]
%     mon_setof(Terme, Buts, Set) :- findall(Terme, Buts, Liste), setof(Terme, Terme^member(Terme, Liste), Set), !.
% \end{lstlisting}